\documentclass{article}
\usepackage{aaryan345h}

\assgnauthor{Aaryan Prakash}
\assgnclass{CS 345H}
\assgntitle{Project Proposal}

\begin{document}
  My project is to implement an interpreter for a simple language 
    that allows sharing values between threads safely using a type system.
  I will be working alone on this project.
  This type system is based on the type system described 
    in the paper ``A Flexible Type System for Fearless Concurrency,''
    which proposes a type system that ensures data races cannot occur
    while being more flexible than Rust's type system.

  In the paper,
    to coordinate access between different threads,
    each thread has some reservation of objects that it can reach.
  The objects itself are represented by a graph of connections.
  A dominating object is one that is on all paths 
    from the root object to all objects reachable from the dominating object.
  If we send the dominating object to another thread,
    then just marking the dominating object as an invalid reference
    is enough to mark the rest of the objects in the subgraph as invalid.

  The steps that I have planned for this project:
  \begin{enumerate}
    \item Implement a basic type system checker
    
      I have not implemented an interpreter with static type checking,
        so implementing a basic one would be useful
        before building a more complex one.
      This would likely be a C-like type system without implicit conversions.
      Essentially,
        each variable has some type,
        and each function call only accepts a certain set of types
        and returns a specific type.
      Types cannot be casted from one type to another implicitly.

      Expected time: 1.5 weeks

    \item Implement global domination invariant

      Certain variables are marked as \verb+iso+,
        which means that they must contain a dominating reference.
      The global domination invariant is that this is required,
        in contrast to the tempered domination invariant described later.

      Expected time: 2.5 weeks

    \item Implement tempered domination invariant 

      This loosens the requirement that a variable marked as \verb+iso+
        must hold a dominating reference at all times.
      Instead,
        the type system can violate this invariant in certain places
        that are tracked by the type system.
      The decision for which references can violate the invariant
        will initially be marked by hand.
      
      Expected time: 2.5 weeks

    \item Create an interpreter for a language using the type system
    
      This interpreter would first run static type checking.
      Once verifying that the types match,
        the interpreter can execute the code.
      The language will be likely be simple,
        with just support for basic types, structs, and functions.
      There would also be some support for basic multithreading,
        similar to how Go uses goroutines for concurrency.

      Expected time: 1.5 weeks
  \end{enumerate}

  Implementing the type system may be a lot of work,
    especially since the paper mentions that their type checking system
    consists 4,100 lines of OCaml.
  In the case when this is too complex to implement,
    I would probably implement a restricted subset of the type system
    or another related type system.
  For example,
    the paper ``Types for Immutability and Aliasing Control''
    describes another type system for sharing values between threads.
  This type system is more restricted than the other type system,
    so it may be easier to implement.

  Another alternative would be to just implement the tempered domination invariant
    without the other additional features brought up in the paper.
  This would be unable to handle certain cases,
    but it would be easier to implement.
  For instance,
    the \verb+if disconnected+ statement is a runtime check
    that also signals information to the type checker.
  Another feature that the paper has
    is that the references that are tracked for tempered domination
    are inferred from the function arguments.
  For these features,
    I will implement the base type checking system first
    and then implement these features.

  References:

  ``A Flexible Type System for Fearless Concurrency,'' ---
  https://www.cs.cornell.edu/andru/papers/gallifrey-types/gallifrey-types.pdf

 ``Types for Immutability and Aliasing Control'' ---
 http://ceur-ws.org/Vol-1720/full5.pdf
\end{document}